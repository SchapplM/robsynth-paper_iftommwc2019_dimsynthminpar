\documentclass{svproc}

\usepackage{amsmath}
\usepackage{amssymb}
\newcommand{\bm}[1]{\boldsymbol{#1}}

% Latex-Makros für häufig verwendete Formelzeichen
\newcommand{\ortvek}[4]{{ }_{(#1)}{\boldsymbol{#2}}^{#3}_{#4} }
\newcommand{\vek}[3]{\boldsymbol{#1}^{#2}_{#3}}
\newcommand{\rotmat}[2]{{{ }^{#1}\boldsymbol{R}}_{#2}}
\newcommand{\rotmato}[2]{{{ }^{#1}\boldsymbol{\overline{R}}}_{#2}}
\newcommand{\transp}[0]{{\mathrm{T}}}
\newcommand{\ks}[1]{{\mathcal{F}}_{#1}}

% Für Deutsche Umlaute
%\usepackage{ngerman}
\usepackage[utf8]{inputenc}

% Für Bilder
\usepackage{graphicx}
\usepackage{color}
\graphicspath{{./figures/}}

% to typeset URLs, URIs, and DOIs
\usepackage{url}
\def\UrlFont{\rmfamily}

\begin{document}
    
\mainmatter              % start of a contribution
%
\title{Exploiting Dynamics Parameter Linearity for Design Optimization in combined Structural and Dimensional Robot Synthesis}
%
\titlerunning{Parameter Linearity in Design Optimization}  % abbreviated title (for running head)
%                                     also used for the TOC unless
%                                     \toctitle is used
%
\author{Moritz Schappler \and Svenja Tappe \and Tobias Ortmaier}
%
\authorrunning{Schappler et al.} % abbreviated author list (for running head)
%
%%%% list of authors for the TOC (use if author list has to be modified)
%\tocauthor{Moritz Schappler, Svenja Tappe, and Tobias Ortmaier}
%
\institute{Institute for Mechatronic Systems, Leibniz University Hannover, Germany,\\
    \email{moritz.schappler@imes.uni-hannover.de}}

\maketitle              % typeset the title of the contribution


%ABSTRACT
\begin{abstract}
...
\end{abstract}

%KEYWORDS
\begin{keywords}
...
\end{keywords}

\section{Introduction}
\label{sec:Intro}

% Was ist Entwurfsoptimierung? Einordnung
%What is design optimization?

% Entwurfsoptimierung wird eingesetzt, um vom Menschen erdachte Maschinen so zu parametrieren, dass sie gestellte Anforderungen bestmöglich erfüllt.

%Can be used to parametrize machines created by a human inventor or to assess the 
% Überblick über Entwurfsoptimierung für Roboter hinsichtlich Antriebsstrang und Segmenten
%Overview Design Optimization for Robots regarding drive train and links

% Eine Bewertung von computergenerierten Strukturen ist damit ebenfalls möglich

The contributions of this paper are
\begin{itemize}
    % Präsentation eines Ansatzes zur schnelleren Berechnung der inversen Dynamik für die Entwurfsoptimierung von Mehrkörpersystemen, insbesondere von Manipulatoren
    % Beispielhafte Rechnungen zum Aufzeigen der Vorteile durch das Verfahren
    % Vorstellung eines Konzepts zur kombinierten Struktur- und Maßsynthese von seriellen und parallelen Robotern
    \item ...
\end{itemize}

The remainder of the paper is structured as follows...

\section{Properties of the Dynamics Model}
\label{sec:DynMdl}

\section{Dimensional Synthesis and Design Optimization}
\label{sec:DimSynth}

...

\section{Exemplary Calculations}

...

\section{Conclusions}
\label{sec:Conclusion}

...

\section*{Acknowledgements}

The financial support from the Deutsche Forschungsgemeinschaft (German Research Foundation, DFG) under grant number OR 196/33-1 is gracefully acknowledged.

% BIBLIOGRAPHY
\bibliographystyle{spmpsci_unsrt}
\bibliography{dimsynth_minpar_ref}

\end{document}
