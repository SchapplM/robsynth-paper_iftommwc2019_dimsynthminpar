\documentclass{svproc}

\usepackage{amsmath}
\usepackage{amssymb}
\newcommand{\bm}[1]{\boldsymbol{#1}}

% Latex-Makros für häufig verwendete Formelzeichen
\newcommand{\ortvek}[4]{{ }_{(#1)}{\boldsymbol{#2}}^{#3}_{#4} }
\newcommand{\vek}[3]{\boldsymbol{#1}^{#2}_{#3}}
\newcommand{\rotmat}[2]{{{ }^{#1}\boldsymbol{R}}_{#2}}
\newcommand{\rotmato}[2]{{{ }^{#1}\boldsymbol{\overline{R}}}_{#2}}
\newcommand{\transp}[0]{{\mathrm{T}}}
\newcommand{\ks}[1]{{\mathcal{F}}_{#1}}

% Für Deutsche Umlaute
%\usepackage{ngerman}
\usepackage[utf8]{inputenc}

% Für Bilder
\usepackage{graphicx}
\usepackage{color}
\graphicspath{{./figures/}}

% to typeset URLs, URIs, and DOIs
\usepackage{url}
\def\UrlFont{\rmfamily}

\begin{document}
    
\mainmatter              % start of a contribution
%
\title{Exploiting Dynamics Parameter Linearity for Design Optimization in combined Structural and Dimensional Robot Synthesis}
%
\titlerunning{Parameter Linearity in Design Optimization}  % abbreviated title (for running head)
%                                     also used for the TOC unless
%                                     \toctitle is used
%
\author{Moritz Schappler \and Svenja Tappe \and Tobias Ortmaier}
%
\authorrunning{Schappler et al.} % abbreviated author list (for running head)
%
%%%% list of authors for the TOC (use if author list has to be modified)
%\tocauthor{Moritz Schappler, Svenja Tappe, and Tobias Ortmaier}
%
\institute{Institute for Mechatronic Systems, Leibniz University Hannover, Germany,\\
    \email{moritz.schappler@imes.uni-hannover.de}}

\maketitle              % typeset the title of the contribution


%ABSTRACT
\begin{abstract}
...
\end{abstract}

%KEYWORDS
\begin{keywords}
...
\end{keywords}

\section{Introduction}
\label{sec:Intro}

% Was ist Entwurfsoptimierung? Einordnung

% Entwurfsoptimierung wird eingesetzt, um vom Menschen erdachte Maschinen so zu parametrieren, dass sie gestellte Anforderungen bestmöglich erfüllen.
% Die Parametrierung kann sich am Beispiel von Robotern auf Kinematik-Parameter wie die Längen der Segmente beziehen, auf die Auslegung der Strukturteile und auf die Auswahl und Auslegung des Antriebsstrangs.


% Abgrenzung Auslegung der Kinematikparameter und Entwurfsoptimierung
%Ansätze für Optimierung der Kinematikparameter
%Mehrkriterielle Optimierung von kinematischen Zielgrößen wie Arbeitsraum und Robotergröße [CeccarelliLan2004] oder zusätzlich Berücksichtigung statischer Zielgröße wie eines Steifigkeitsmodells für serielle und parallele Roboter [CarboneOttCec2007] oder dynamischer Zielgrößen, wie der über die Trajektorie verbrauchten Energie [RamirezKotOrt2017]
%Die Entwurfsoptimierung bezüglich kinematischer Parameter kann auch gemeinsam mit die Dynamik beeinflussenden Parametern wie eines parametrischen CAD-Modells  [TarkianLunÖl2008,ZhouBai2015] oder der Antriebsstrangauslegung [ShillerSun1991] erfolgen

% Überblick über Entwurfsoptimierung für Roboter hinsichtlich Antriebsstrang und Segmenten

%Auslegung des Antriebsstrangs:
% Grundprinzip ist eine gegebene Roboterstruktur mit definierten Kinematikparameter, dessen Antriebe an eine gegebene Trajektorie oder Punkt-zu-Punkt-Bewegung angepasst werden \cite{ChedmailGau1990}
% Zielfunktion der Optimierung ist eine kurze Zykluszeit \cite{TarkianPerOelFen2011}, Masse der Antriebe \cite{ChedmailGau1990,PetterssonOel2009,ZhouBaiHan2011}, Lebensdauer der Getriebe \cite{PetterssonAndKru2005}, monetäre Kosten \cite{PetterssonAndKru2005}

%Nebenbedingungen sind die Bemessungsmomente und Drehzahl von Motor und Getriebe [ChedmailGau1990,PetterssonOel2009,ZhouBaiHan2011], 

% Mögliche zusätzliche Erweiterungen der Simulation des Systems sind eine zusätzliche Dynamik aus einem thermischen Motormodell [ChedmailGau1990] oder die Berechnung der Vorwärtsdynamik des geregelten Systems mit Betrachtung des Regelfehlers und der elektrischen Motordynamik [Padilla-GarciaCruRod2015], anstelle der reinen Betrachtung der inversen Dynamik mit Vernachlässigung regelungstechnischer Effekte.


% Auslegung der Struktur
% Bei Optimierung der Strukturparameter kann ebenfalls deren Masse als Zielfunktion verwendet werden [TarkianPerOelFen2011]

%[ZhouBai2015]

% Die Ansätze können weiterhin darin unterschieden werden, ob die Trajektorie im Zeitverlauf Teil der Optimierung ist [PetterssonAndKru2005], oder ob diese als konstant gegeben ist [TarkianPerOelFen2011,ZhouBaiHan2011,Padilla-GarciaCruRod2015].

% Bei einer zusätzlichen Optimierung der Segmentlängen ist aufgrund der Neuberechnung der inversen Kinematik auch die Neuberechnung der Trajektorie erforderlich [TarkianLunÖl2008,ZhouBai2015]


% Der Optimierungsalgorithmus zur Berücksichtigung der diskreten Entwurfsvariablen sind genetische Algorithmen [TarkianPerOelFen2011,Padilla-GarciaCruRod2015], Particle Swarm Optimierung [RamirezKotOrt2017] oder Complex-Algorithmus mit Modifikationen bezüglich der Rundung und Interpolation in der Auswahltabelle  [PetterssonAndKru2005] [ZhouBaiHan2011] oder die Interpolation der Auswahltabelle und Überführung in ein kontinuierliches Problem [ChedmailGau1990]
% Je nach Art der Annahmen über die Antriebe und Segmente ist ein rekursives Vorgehen von distalen zu proximalen Gelenken möglich [ChedmailGau1990].
% ...zur Berücksichtigung  der diskreten Entwurfsvariablen Complex, PSO, GA
% ...sind Gradientenbasierte Verfahren [], Simplex [TarkianPerOelFen2011]

% ...Eine Bewertung von computergenerierten Strukturen ist damit ebenfalls möglich

% Kurzfassung des Beitrags des Papers
% Alle bisherigen Arbeiten zur Entwurfsoptimierung vernachlässigen die Eigenschaft der Parameterlinearität der Dynamik-Gleichungen, die hauptsächlich aus dem Bereich Parameteridentifikation bekannt ist.
%[ChedmailGau1990] nennt zwar die Nutzung der parameterlinearen Form zur Berechnung der Dynamik und geht auf die Anpassung der Parameter ein, die Ausnutzung der Parameterlinearität in der Entwurfsoptimierung wird aber nicht erwähnt. Da auch keine der folgenden Arbeiten darauf eingehen, wird dieser Aspekt in diesem Paper hervorgehoben.

The contributions of this paper are
\begin{itemize}
    % Präsentation eines Ansatzes zur schnelleren Berechnung der inversen Dynamik für die Entwurfsoptimierung von Mehrkörpersystemen, insbesondere von Manipulatoren
    % Vorstellung eines Konzepts zur kombinierten Struktur- und Maßsynthese von seriellen und parallelen Robotern, Darstellung der Annahmen für die schnellere Berechnung
    % Beispielhafte Rechnungen zum Aufzeigen der Vorteile durch das Verfahren
    \item ...
\end{itemize}

The remainder of the paper is structured as follows...

% * Vorstellung des Konzepts der kombinierten Struktur- und Maßsynthese und Herleitung des Bedarfs einer effizienten Entwurfsoptimierung
% * Zeigen der Eigenschaften der verwendeten Dynamikmodellierung 
% * Einsatz dieser Dynamikmodellierung in der Entwurfsoptimierung
% * Überschlägige Rechnung zum Potential der Verbesserung
% * Zusammenfassung

\section{Combined Structural and Dimensional Synthesis}
\label{sec:DimSynth}

%[RamirezKotOrt2015]
% Idee der kombinierten Struktur- und Maßsynthese: Für eine gegebene Roboteraufgabe mit Freiheitsgraden und Trajektorie und Last werden alle geeigneten seriellen und parallelen Strukturen ermittelt und davon die bestgeeignesten ausgewählt.
% Dadurch wird der Vergleich einer Vielzahl von Strukturen notwendig. Z.B. 10 für 2T1R, 35 für 3T1R, 326 für 3T3R serielle Strukturen.
% Für Parallele Strukturen ergeben sich ebenfalls eine Vielzahl an Kombinationen, wie z.B. von Gogu mit Hilfe der evolutionären Morphologie ermittelt [Gogu].

%Dadurch, dass mehrere verschachtelte Schleifen durchlaufen werden ist die Rechenzeit relativ hoch. Besonders der Rechenaufwand für die innere Schleife ist entscheidend für die Gesamtlaufzeit des Algorithmus
%Nennung der Schleifen: Über alle Roboter, Über alle Kinematikparameter, Antriebs- und Segmentauslegung basierend auf inverser Dynamik zu allen Zeitschritten der Trajektorie
% Blockschaltbild mit ungefährem Gesamtablauf
% Für die Berechnung Dynamischer Leistungsmerkmale wie Gesamtmasse oder Energie (basierend auf den Gelenkmomenten) ist die Berechnung der inversen Dynamik notwendig.

% Damit der Vergleich zwischen stark verschiedenen Strukturen mit möglichst wenigen Kennzahlen möglich ist:
% Annahme für die Maßsynthese: Trajektorie bleibt gleich
% Ansonsten müsste beim Vergleich der Strukturen verschiedene Kennzahlen wie Arbeitsraum, Fußabdruck, Energieverbrauch und zusätzlich die Verfahrdauer der Trajektorie einkriteriell gewichtet oder mehrkriteriell z.B. mittels Pareto-Front vergleichen werden.
%Durch die Vorgabe der Trajektorie wird die Komplexität reduziert und die im Folgenden dargestellte Methode wäre nicht möglich.

\section{Properties of the Dynamics Model}
\label{sec:DynMdl}

%[KhalilDom2002]
% Dynamikmodell allgemein in verallgemeinerten Koordinaten, Grundlage für Antriebsauslegung (Modell mit Antrieb?)
% Parameterlineare Form der Dynamik
% Prinzip der Parametergruppierung: Zusammenfassung von Parametern vom Endeffektor zur Basis
% Eigenschaft der Regressorform: Obere rechte Dreiecksform; die Gelenkmomente der letzten Achsen hängen nur von den letzten Dynamikparametern ab, die Gelenkmomente der ersten Achse hängen von allen Parametern ab. Dünne Besetztheit der Regressormatrix kann genutzt werden.

% Für Segmentauslegung: Gelenkmomente sind allein nicht ausreichend. Bei Verwendung der Schnittkräfte in den Gelenken vor und nach dem Segment ist ebenfalls eine parameterlineare Darstellung möglich, allerdings sind durch Parameterzusammenfassung keine oder wesentlich weniger Vereinfachungen möglich als bei alleiniger Betrachtung der Gelenkmomente.

\section{Improve the Efficiency of Design Optimizing}
\label{sec:DesOptImprove}

% Formel: Optimierungsproblem
% Formel: Ausklammern des Regressors aus einer Operation
% Blockschaltbild Optimierung ohne Ausnutzung der Parameterlinearen Form
% Blockschaltbild Optimierung mit Ausnutzung

%Rekapitulation der Annahme: Ist nicht möglich, wenn die Kartesische Trajektorie verändert wird und indirekt Zielfunktion der Optimierung ist [TarkianÖlFenPet2009,PetterssonOel2009,TarkianPerOelFen2011] oder Nebenbedingung [PetterssonAndKru2005] ist oder wenn die Kinematischen Längen Teil der Optimierung sind [TarkianLunÖl2008,ZhouBai2015]
% ...oder die Nebenbedingungen beeinflusst.
...

\section{Exemplary Calculations}

% Tabelle mit EE-FG, Anzahl Gelenke, Anzahl Kinematikparameter, Dimension Regressor, Anzahl Nullen, Operationen Dynamik aus Regressor, Operationen Matrix-Multiplikation.
% Für Drei Roboter: LBR, Kuka 6FG, SCARA
% Aussage: Bei Verwendung des Ausklammerns der Dynamikgleichungen kann man ...% der Rechenzeit sparen.
...

\section{Conclusions}
\label{sec:Conclusion}

% Vorstellung des Verfahrens zur Ausnutzung ...
% Ausblick: Anwendung des Verfahrens zum Vergleich von seriellen und parallelen Robotern in der kombinierten Struktur- und Maßsynthese.

...

\section*{Acknowledgements}

The financial support from the Deutsche Forschungsgemeinschaft (German Research Foundation, DFG) under grant number OR 196/33-1 is gracefully acknowledged.

% BIBLIOGRAPHY
\bibliographystyle{spmpsci_unsrt}
\bibliography{dimsynth_minpar_ref}

\end{document}
