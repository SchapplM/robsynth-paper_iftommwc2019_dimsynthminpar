% !TEX program = lualatex
\documentclass[c]{beamer}

%% Possible paper sizes: a0, a0b, a1, a2, a3, a4.
%% Possible orientations: portrait, landscape
%% Font sizes can be changed using the scale option.
\usepackage[size=a0,orientation=portrait,scale=1.25]{beamerposter}
\setbeamersize{text margin left=25mm,
               text margin right=25mm}

\usetheme{imes-poster}
\usecolortheme{LUH}
\usefonttheme{professionalfonts}

\usepackage[english]{babel}

\usepackage{multicol}
\setlength\columnsep{2cm}
\usepackage{blindtext}
\usepackage{amsmath}
\usepackage{amssymb}
%\newcommand{\bm}[1]{\boldsymbol{#1}}
\usepackage{bm}
\graphicspath{{./figures/}}

\usepackage[no-math]{fontspec}
\setmainfont[
    ]{Rotis Sans Serif Std}
    
% Different math fonts

%\usepackage{mathpazo}
%\usepackage{sfmath}
\usepackage{newtxsf}
    
\author[moritz.schappler@imes.uni-hannover.de]{Moritz Schappler}
\title{Exploiting Dynamics Parameter Linearity for Design Opti-\\mization in a Combined Structural and Dimensional Synthesis}
\institute{Institute of Mechatronic Systems}
%\newcommand{\researchfield}{Medical Imaging}
\newcommand{\firstauthorline}{Moritz Schappler, M.\,Sc.}
\newcommand{\secondauthorline}{Svenja Tappe, M.\,Sc.,}
\newcommand{\thirdauthorline}{Prof.\,Dr.-Ing. Tobias Ortmaier}
%\newcommand{\firstaffiliation}{Affiliation first line}
%\newcommand{\secondaffiliation}{Affiliation second line}
%\newcommand{\thirdaffiliation}{Affiliation third line}

\newcommand{\postersubsection}[1]{%
\setlength\fboxsep{0pt}%
\vfil\penalty125\vfilneg\vskip1.5ex
\colorbox{Grau}{\parbox[b]{\columnwidth}{\vskip0.75ex%
\Large\hskip1ex #1%
\vskip0.75ex}}%
}

\renewcommand{\arraystretch}{1.2} % more vertical padding for tabular

\begin{document}
\begin{frame}
%%%%%%%%%%%%%%%%%%%%%%%%%%%%%%%%%%%%%%%%%%%%%%%%%%%%%
\begin{block}{Combined Structural and Dimensional Robot Synthesis}
\parbox{\linewidth}{
\begin{multicols}{2}
%[
%]
\begin{itemize}
    \item Motivation
    \item Question to be answered
    \item Combined structural and dimensional synthesis
    \item Result of structural synthesis, how to use it?
\end{itemize}

\begin{figure}[t]
    \centering
	\input{./figures/comb_struct_dim_synth.pdf_tex}
%    \caption{Delicious caffe.}
\end{figure}


%\vspace{30cm}

%\postersubsection{Überschrift 2}
%\begin{figure}[t]
%    \centering
%    \includegraphics[width=\columnwidth,trim=0 0cm 0 0cm, clip]{Struktursynthese_Fragezeichen.pdf}
%%    \caption{Delicious caffe.}
%\end{figure}

\end{multicols}}
\end{block}
%%%%%%%%%%%%%%%%%%%%%%%%%%%%%%%%%%%%%%%%%%%%%%%%%%%%%
\begin{whiteblock}{Dimensional Synthesis and Robot Model}
\parbox{\linewidth}{
\begin{multicols}{2}
%[
%All human things are subject to decay. And when fate summons, Monarchs must obey. This sentence can be even longer!
%]
\begin{itemize}
    \item 1
    \item 1
    \item 1
    \item 1
    \item 1
\end{itemize}
%\begin{figure}[t]
%    \centering
%    \includegraphics[width=\columnwidth,trim=0mm 0mm 0mm 0cm, clip]{./plot.pdf}
%    \caption{$ \text{tanh} \left( x \right) $ and its derivatives.}
%    \label{fig:2}
%\end{figure}
\begin{figure}[t]
    \centering
    \input{./figures/dim_synth_dynamics.pdf_tex}
    %    \caption{Delicious caffe.}
\end{figure}

\end{multicols}}
\end{whiteblock}
%%%%%%%%%%%%%%%%%%%%%%%%%%%%%%%%%%%%%%%%%%%%%%%%%%%%%
\begin{block}{Results}
\parbox{\columnwidth}{
\begin{multicols}{2}
[]
\begin{itemize}
    \item One
    \item Two
    \item Three
\end{itemize}

%\postersubsection{Überschrift 2}

\begin{figure}[t]
    \centering
    \input{./figures/reglin_results.pdf_tex}
    %    \caption{Delicious caffe.}
\end{figure}


\end{multicols}}
\end{block}
%%%%%%%%%%%%%%%%%%%%%%%%%%%%%%%%%%%%%%%%%%%%%%%%%%%%
\end{frame}
\end{document}
