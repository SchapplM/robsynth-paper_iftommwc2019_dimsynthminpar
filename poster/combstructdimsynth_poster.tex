% !TEX program = lualatex
\documentclass[c]{beamer}

%% Possible paper sizes: a0, a0b, a1, a2, a3, a4.
%% Possible orientations: portrait, landscape
%% Font sizes can be changed using the scale option.
\usepackage[size=a0,orientation=portrait,scale=1.25]{beamerposter}
\setbeamersize{text margin left=25mm,
               text margin right=25mm}

\usetheme{imes-poster}
\usecolortheme{LUH}
\usefonttheme{professionalfonts}

\usepackage[english]{babel}

\usepackage{multicol}
\setlength\columnsep{2cm}
\usepackage{blindtext}
\usepackage{amsmath}
\usepackage{amssymb}
%\newcommand{\bm}[1]{\boldsymbol{#1}}
\usepackage{bm}
\graphicspath{{./figures/}}

\usepackage[no-math]{fontspec}
\setmainfont[
    ]{Rotis Sans Serif Std}
    
% Different math fonts

%\usepackage{mathpazo}
%\usepackage{sfmath}
\usepackage{newtxsf}
    
\author[moritz.schappler@imes.uni-hannover.de]{Moritz Schappler}
\title{Exploiting Dynamics Parameter Linearity for Design Opti-\\mization in a Combined Structural and Dimensional Synthesis}
\institute{Institute of Mechatronic Systems}
%\newcommand{\researchfield}{Medical Imaging}
\newcommand{\firstauthorline}{Moritz Schappler, M.\,Sc.}
\newcommand{\secondauthorline}{Svenja Tappe, M.\,Sc.,}
\newcommand{\thirdauthorline}{Prof.\,Dr.-Ing. Tobias Ortmaier}
%\newcommand{\firstaffiliation}{Affiliation first line}
%\newcommand{\secondaffiliation}{Affiliation second line}
%\newcommand{\thirdaffiliation}{Affiliation third line}

\newcommand{\postersubsection}[1]{%
\setlength\fboxsep{0pt}%
\vfil\penalty125\vfilneg\vskip1.5ex
\colorbox{Grau}{\parbox[b]{\columnwidth}{\vskip0.75ex%
\Large\hskip1ex #1%
\vskip0.75ex}}%
}

\renewcommand{\arraystretch}{1.2} % more vertical padding for tabular

\begin{document}
\begin{frame}
%%%%%%%%%%%%%%%%%%%%%%%%%%%%%%%%%%%%%%%%%%%%%%%%%%%%%
\begin{block}{Combined Structural and Dimensional Robot Synthesis}
\parbox{\linewidth}{
\begin{multicols}{2}
%\postersubsection{Motivation}
\begin{itemize}
    \item Task specific robots allow improvement of performance criteria such as energy consumption compared to standard industrial robots.
%\end{itemize}
%\postersubsection{Motivation}
%\begin{itemize}
    \item Structural synthesis: ``Which kinematic structures have the required degrees of freedom for my task?'' 
    \begin{itemize}
    \item result: joint types and alignment, leg chain orientation for parallel robots
    \end{itemize}
    \item Dimensional synthesis: ``What are the optimal dimensions for the one robot structure at hand?'' 
    \begin{itemize}
        \item results: link lengths, link strength, choice of drive train components, choice of passive joints
    \end{itemize}
    \item Combined synthesis: ``Which robot is the best for my task?''
    \begin{itemize}
        \item requires calculation of the performance criteria for all possible robot structures
    \end{itemize}
    \item High computational effort: Dimensional synthesis for all results of the structural synthesis. Requires improved efficiency.
%    \item text here
%    \item text here
\end{itemize}

\begin{figure}[t]
    \centering
	\input{./figures/comb_struct_dim_synth.pdf_tex}
\end{figure}


\end{multicols}}
\end{block}
%%%%%%%%%%%%%%%%%%%%%%%%%%%%%%%%%%%%%%%%%%%%%%%%%%%%%
\begin{whiteblock}{Dimensional Synthesis and Dynamics Parameter Linearity}
\parbox{\linewidth}{
\begin{multicols}{2}
\begin{itemize}
    \item Assumption: Task trajectory $\bm{x}(t)$ is given and not part of optimization
    \item Allows exploiting the linearity of the rigid body dynamics
\end{itemize}
\begin{align}
\bm{\tau}(\bm{q},\dot{\bm{q}},\ddot{\bm{q}},\bm{\varrho},\bm{\delta})
&= 
\bm{M}(\bm{q},\bm{\varrho},\bm{\delta}) \ddot{\bm{q}}
+ \bm{c}(\bm{q},\dot{\bm{q}},\bm{\varrho},\bm{\delta})
+ \bm{g}(\bm{q},\bm{\varrho},\bm{\delta}) \nonumber \\
&=
\bm{\varPhi}_{\bm{\tau}}(\bm{q},\dot{\bm{q}},\ddot{\bm{q}},\bm{\varrho}) \bm{\delta} \nonumber
\end{align}
Multiple nested Optimization Loops to find values for the parameters:
\begin{itemize}
    \item \emph{Dimensional synthesis}: Find kinematic parameters $\bm{\varrho}$ based on global criteria (e.\,g. energy consumption) and global constraints (e.\,g. minimum workspace size)
    \item \emph{Design optimization} of drive trains and links gives dynamics parameters $\bm{\delta}$. Uses global criteria from the dimensional synthesis and intermediate criteria (e.\,g. minimal mass) and constraints (torque and velocity limits, joint range)
\end{itemize}
Possible improvements for serial and parallel robots:
\begin{itemize}
    \item Simplification of the calculation by exploiting the parameter linearity of the dynamics (lower part of the figure)
%    \item Possible for serial link and parallel robots.
    \item Prerequisite: General inverse kinematics and dynamics modeling for all robot structures (serial/parallel) from the structural synthesis
    \begin{itemize}
        \item Gradient-based inverse kinematics and projection-based dynamics for parallel robots
    \end{itemize}
    \item Serial robots: Recursive optimization from distal to proximal links
    \item Parallel robots: Combined design optimization of all parameters
%    \item text here
%    \item text here
\end{itemize}
%\begin{figure}[t]
%    \centering
%    \includegraphics[width=\columnwidth,trim=0mm 0mm 0mm 0cm, clip]{./plot.pdf}
%    \caption{$ \text{tanh} \left( x \right) $ and its derivatives.}
%    \label{fig:2}
%\end{figure}
\begin{figure}[t]
    \centering
    \input{./figures/dim_synth_dynamics.pdf_tex}
    %    \caption{Delicious caffe.}
\end{figure}

\end{multicols}}
\end{whiteblock}
%%%%%%%%%%%%%%%%%%%%%%%%%%%%%%%%%%%%%%%%%%%%%%%%%%%%%
\begin{block}{Results and Ongoing Works}
\parbox{\columnwidth}{
\begin{multicols}{2}
[]
\begin{itemize}
    \item Serial Robots: Time reduction of inverse dynamics about 70\,\%
    \begin{itemize}
        \item Robots: SCARA (4-DoF), industrial robot (6-DoF), collaborative lightweight robot (7-DoF)
%        \item Left: Read: For changing the dynamics parameters 10 times the calculation for the LWR trajectory takes 0.9\,s with the standard method and 0.5\,s with the proposed method.
        \item Left: Changing the parameters 10 times reduces the time for the LWR from 0.9\,ms to 0.5\,ms.
        \item Right (only LWR): Using just 10 robot poses reduces the time only about 20\,\%.
    \end{itemize}
    \item Recursion for serial robots: dynamics calculation only small part of the design optimization, therefore less overall benefit 
    \item Higher potential for parallel robots: Dynamics re-evaluation necessary for each parameter iteration (ongoing works)
\end{itemize}

\begin{figure}[t]
    \centering
    \input{./figures/outlook_serial2parallel.pdf_tex}
\end{figure}

%\postersubsection{Überschrift 2}

\begin{figure}[t]
    \centering
    \input{./figures/reglin_results.pdf_tex}
\end{figure}


\end{multicols}}
\end{block}
%%%%%%%%%%%%%%%%%%%%%%%%%%%%%%%%%%%%%%%%%%%%%%%%%%%%
\end{frame}
\end{document}
