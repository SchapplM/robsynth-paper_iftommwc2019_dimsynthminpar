% !TEX program = lualatex
\documentclass[c]{beamer}
%% Possible paper sizes: a0, a0b, a1, a2, a3, a4.
%% Possible orientations: portrait, landscape
%% Font sizes can be changed using the scale option.
\usepackage[size=a0,orientation=portrait,scale=1.25]{beamerposter}
\setbeamersize{text margin left=25mm,
               text margin right=25mm}

\usetheme{imes-poster}
\usecolortheme{LUH}

\usepackage[english]{babel}
\usepackage[libertine]{newtxmath}
\usepackage[no-math]{fontspec}
\usepackage{multicol}
\setlength\columnsep{2cm}
\usepackage{blindtext}

\setsansfont[
    BoldFont={Rotis Sans Serif Std Bold}, 
    ItalicFont={Rotis Sans Serif Std Italic},
    ]{Rotis Sans Serif Std}
    
\author[author@imes.uni-hannover.de]{Main Author}
\title{Yet Another beamerposter Theme\\ (Variable Sizes and Colour Themes)}
\institute{Institut of Mechatronic Systems}
\newcommand{\researchfield}{Medical Imaging}
\newcommand{\firstauthorline}{Authors first line,}
\newcommand{\secondauthorline}{Authors second line,}
\newcommand{\thirdauthorline}{Authors third line}
\newcommand{\firstaffiliation}{Affiliation first line}
\newcommand{\secondaffiliation}{Affiliation second line}
\newcommand{\thirdaffiliation}{Affiliation third line}

\newcommand{\postersubsection}[1]{%
\setlength\fboxsep{0pt}%
\vfil\penalty125\vfilneg\vskip1.5ex
\colorbox{Grau}{\parbox[b]{\columnwidth}{\vskip0.75ex%
\Large\hskip1ex #1%
\vskip0.75ex}}%
}

\renewcommand{\arraystretch}{1.2} % more vertical padding for tabular

\begin{document}
\begin{frame}
%%%%%%%%%%%%%%%%%%%%%%%%%%%%%%%%%%%%%%%%%%%%%%%%%%%%%
\begin{block}{Überschrift 1}
\parbox{\linewidth}{
\begin{multicols}{2}
[
]
\blindtext[2]

\postersubsection{Überschrift 2}
%\begin{figure}[t]
%    \centering
%    \includegraphics[width=\columnwidth,trim=0 6cm 0 2cm, clip]{./caffe.jpg}
%    \caption{Delicious caffe.}
%    \label{fig:1}
%\end{figure}

\end{multicols}}
\end{block}
%%%%%%%%%%%%%%%%%%%%%%%%%%%%%%%%%%%%%%%%%%%%%%%%%%%%%
\begin{whiteblock}{Überschrift 1}
\parbox{\linewidth}{
\begin{multicols}{3}
[
All human things are subject to decay. And when fate summons, Monarchs must obey. This sentence can be even longer!
]
\blindtext[1]
%\begin{figure}[t]
%    \centering
%    \includegraphics[width=\columnwidth,trim=0mm 0mm 0mm 0cm, clip]{./plot.pdf}
%    \caption{$ \text{tanh} \left( x \right) $ and its derivatives.}
%    \label{fig:2}
%\end{figure}
\blindtext[1]
\end{multicols}}
\end{whiteblock}
%%%%%%%%%%%%%%%%%%%%%%%%%%%%%%%%%%%%%%%%%%%%%%%%%%%%%
\begin{block}{Überschrift 1}
\parbox{\columnwidth}{
\begin{multicols}{2}
[]
\begin{itemize}
    \item One
    \item Two
    \item Three
\end{itemize}

It is well known that
\begin{equation}
e^{j \varphi} = \cos(\varphi) + j \sin(\varphi).
\end{equation}
Another famous inline equation is $ \sqrt{2} = 1.41421356237\!\ldots $

\postersubsection{Überschrift 2}

\blindtext

\begin{table}
    \begin{tabular}{cccc}
        \hline
        first & second & third & fourth \\
        \hline
        1 & 2 & 3 & 4 \\
        5 & 6 & 7 & 8 \\
        \hline
    \end{tabular}
    \caption{Fancy numbers.}
\end{table}

\end{multicols}}
\end{block}
%%%%%%%%%%%%%%%%%%%%%%%%%%%%%%%%%%%%%%%%%%%%%%%%%%%%%
\end{frame}
\end{document}
